\section{Modeling Combinatorial Objects}
\label{sec:modeling}


A combinatorial object is an instance of 
a {\em type of combinatorial objects}.
A combinatorial object lives in a space with symmetry and there is a notion of equivalence.
Here, we wish to describe combinatorial objects using functions. The equivalence of combinatorial objects 
translates to a notion of equivalence of these functions. 
There are two basic models. One model consists of functions between sets. 
The second model has linear functions beween vector spaces. 
We call the two models the {\em set-model} and the 
{\em linear model}.
We will first introduce the two models and then discuss some examples of combinatorial objects and 
how they can be modeled.


\bigskip


The set-model consists of three ingredients. There is 
\begin{enumerate}
\item a group $G$ acting on a finite set $D$,
\item a group $H$ acting on a finite set $R$, 
\item a set $\frakF$ of functions from $D$ to $R$.
\end{enumerate}
Two functions $f_1,f_2 \in \frakF$ are equivalent if there exist elements 
$g \in G$ and $h \in H$ such that 
$$
\bar{h} \circ f_1 \circ \bar{g}^{-1} = f_2 
$$
Here, the bar signifies that we consider the mappings which are induced by the 
respective element under the group action.
This definition defines an action of $G \times H$ on $\frakF.$

\bigskip

The {\em linear model} consists of 
\begin{enumerate}
\item a finite field $\bbF_q,$
\item vector spaces $D$ and $R$ over $\bbF_q$,
\item a semilinear group $G$ acting on a finite vector space $D$,
\item a semilinear group $H$ acting on a finite vector space $R$, 
\item a set $\frakF$ of $\bbF_q$-linear mappings from $D$ to $R$.
\end{enumerate}
Two linear functions $f_1,f_2 \in \frakF$ are equivalent if there exist elements 
$g \in G$ and $h \in H$ such that 
$$
\bar{h} \circ f_1 \circ \bar{g}^{-1} = f_2 
$$


%Since linear mappings are defined once we know the image on a basis, 
%the elements of $\frakF$ can be identified with $k \times n$ matrices over $\bbF_q$
%(using the convention from coding theory that matrices act on the right). 




%The construction rule governs 
%the assembly of the instances of combinatorial objects of this type.
%We are allowed to form subsets, 
%sequences, subsets of a fixed size, etc. We are also 
%able to stack sets and sequences 
%into larger sets and sequences. 
%Stacked objects can be stacked in turn.
%There is no limit on the stacking depth 
%except that it needs to be finite.
%We can think of this part of the description as an analogue of 
%a data structure in computer science.




\bigskip

%We say that we consider objects 
%{\em of a certain type} if the family $\frakF$ has been chosen.
The notion of equivalence introduces an 
equivalence relation on the set of combinatorial objects 
of a fixed type.
It is also customary to call two objects {\em isomorphic} if they are equivalent.
In Kerber~\cite{Kerber99}, the orbits of the group $G \times H$ on $\frakF$ under the action introduced above 
are called {\em symmetry classes of mappings}.
The {\em isomorphism class} of a mapping $f \in \frakF$ is the set 
of all $f' \in \frakF$ which are isomorphic to $f$.


\bigskip



Combinatorial objects often come in families.
A {\em family of combinatorial objects} is a sequence 
of combinatorial objects where the construction rule 
is fixed and only the basic set 
and the group action vary with a parameter $n$, say. 
There may be restrictions on the paramater $n$. 
For a family, we require that there are infinitely 
many possible values of $n$. 




\bigskip





The task of determining the isomorphism classes of a certain 
combinatorial object is 
known as the {\em classification problem}.
We distinguish different levels of solving 
the classification problem. 
There is a hierarchy of problems, listed here in the order of increasing difficulty:
\begin{enumerate}
\item
The {\em existence problem}: 
We ask if there exists a combinatorial object 
of a given type. 
\item
The {\em orbit counting problem}:
We determine 
the number of isomorphism types 
of combinatorial objects of a given type.
At this level, we will be satisfied with the number of orbits.
This problem is treated in enumerative combinatorics.
\item
The {\em constructive enumeration / classification problem}: 
We ask to determine representatives for each 
isomorphism class.
\end{enumerate}


\bigskip

%What does it mean that we solve the classification problem 
%for a type of combinatorial object?
%Recall that combinatorial objects often come in families, 
%parameterized by some integer $n$, say. 
%Solving the classification problem 
%for the family means solving the classification problem 
%for each legal parameter value $n$.
%For many families of combinatorial objects, the classification is out of reach.
%Nevertheless, it is often helpful to solve the classification problem 
%for small instances of family members. 
%This may give insight into the general problem. 



%\bigskip





A classification problem as described above 
aims at finding a system 
of representatives for the orbits of a group $G$
on a specific set of objects. 
A system of representatives consists of exactly one 
object for each of the different $G$-orbits.
Such a system is called a {\em transversal of the orbits.}
Together with the representatives, we also want to 
find the stabilizer groups and the orbit lengths. 

\bigskip

Beyond finding the transversal, 
we also want to be able to {\em recognize} objects and identify orbits.
This means the following:
Given any object, we want to determine the unique 
element in the transversal isomorphic to the given object.
This element is called the {\em orbit representative} 
associated to the given object.
The {\em recognition problem} requires 
us to recognize objects. 
A refined version of this is the 
{\em constructive recognition problem}. It requires 
us to recognize orbits 
and to identify a group element $g \in G$ which maps the 
given object to the representative.


\bigskip


Let us look at some combinatorial objects and how 
to represent them as symmetry classes of mappings.

\begin{enumerate}
\item
{\bf Graphs}: A graph on $n$ vertices is a binary relation on a set 
$V$ of $n$ elements, called vertices. 
The pairs of vertices which are incident are called edges. 
The edge set $E$ is the set of all edges. The graph is thus a pair $(V,E)$. 
For simple graphs, we require that the binary relation is 
totally irreflexive and symmetric. 
Let the group $G = \Sym_n$ act on the set $V = \bbZ_n.$ Let $D$ be the set of 
unordered pairs of elements from $V$. Then $G$ acts on $D$ also. 
Let $R = \bbZ_2$ and consider the trivial group $H=1$ acting on $R$. 
Let $\frakF$ be the class of all functions from $D$ to $R$.
Then $\frakF$ under the action of $G \times H = G \times 1$ 
corresponds to the simple graphs on $n$ vertices.
\item
{\bf Sets of size $k$ under equivalence}: 
Many combinatorial objects arise naturally as orbits of a group acting on 
$k$-subsets of a fixed set, for some fixed $k$. 
Let $X$ be a set and let $G$ be a group acting on $X.$ 
The $G$-orbits on $k$-subsets of $X$ can be described as the set $\frakF$ 
of symmetry classes of mappings $f: \bbZ_k \rightarrow X$ 
under the action of $\Sym_k \times G.$ 
The groups are $\Sym_k$ acting on $D = \bbZ_k$ and $G$ acting on $R=X$.
\item
{\bf Subspaces of dimension $k$ in $\bbF_q^n$ under equivalence}: 
Consider a group $A \le \Gamma {\rm L}(n,q)$ 
acting on subspaces of dimension $k$ for some $k \le n.$
The orbits of this action can be described as symmetry classes of mappings. 
Pick $D = \bbF_q^k$ and $R = \bbF_q^n.$ 
Furthermore, pick $G = {\rm GL}(k,q)$ and $H = A$.
The functions which we consider are the $\bbF_q$-linear functions $f : D \rightarrow R$ 
which are one-to-one. Thus
$$
\frakF = \{ f : D \rightarrow R \mid f \; \mbox{is linear and one-to-one}\; \}. 
$$
\end{enumerate}


