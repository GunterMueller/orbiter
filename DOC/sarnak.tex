\subsection{Ramanujan Graphs}


An interesting family of graphs was introduced by Lubotzky, Phillips and Sarnak in 1988. 
These are Ramanujan graphs. The graphs need a choice of two primes, $p$ and $q$, 
both of which are required to be congruent 1 modulo 4.
The graphs are created as Cayley graphs inside a certain group $G$.
Let $G$ be a group and let $S$ be a subset of $G.$ 
We require that $S$ is closed under inverses. 
The cayley graph $\Cay(G,S)$ is defined as follows:
The vertices of the graph are the elements of the group $G$. 
The edges are determined by the set $S$. 
Namely, 
a vertex $g$ is adjacent to a vertex $h$ if there is an element $s \in S$ with $h=gs.$ 
Observe that since vertices are associated with elements in $G$, the equation $h=gs$ is computed inside the group $G$.
The condition that $S$ is closed under inverses means that $s^{-1} \in S$ also, 
and hence that $h$ is adjacent to $g$ since
$hs^{-1} =g$. Thus, the edges in the Caley graph are well defined.
The set $S$ is called the connection set, since it gives the neighbors of the vertex associated to the identity 
element. 
Every other vertex looks that same. The reason for this is that the group $G$ 
acts as a group of automorphisms on the graph. This action is vertex transitively. 
In the example described by Lubotzky, Phillips and Sarnak, 
the group $G$ is either $\PGL(2,q)$ or $\PSL(2,q)$, 
depending on the value of the Legendre symbol for $p$ and $q.$
The connection set $S$ is derived from the set of $p+1$  solutions 
of $a_0^2+a_1^2+a_2^2+a_3^2 = p$ over integers $(a_0,a_1,a_2,a_3).$
The paper shows that each vertex is incident with $p+1$ other vertices.
The example shown here demonstrates that this is slightly incorrect.
For $p=29$ and $q=5$, we get a graph where each vertex is incident with only $27$ vertices.


{\small
{\tt
\input CODE/sarnak.tex
}
}

%Here is the output of the program for $q=5$ and $p=29$:


%{\tt
%\input CODE/sarnak_out.tex
%}

Here is a drawing of the graph created for $q=5$ and $p=29$.
The graph has $60$ vertices which are in correspondence to the elements of $\PSL(2,5).$
Each vertex is incident with $27$ vertices.
%The left picture shows the edges incident with the first vertex only, 
%the second drawing shows the full graph:
$$
%\input GRAPHICS/X_29_5_one.tex
%\quad
\input GRAPHICS/X_29_5.tex
$$


\bigskip


