
\section{Examples}
\label{sec:examples}

Let us illustrate this technique of classifying combintorial objects. 
We go back to the examples already introduced.

\bigskip

Let us first consider the problem of classifying graphs with $n$ vertices. 
Let 
$V = \{0,\ldots,n-1\}$ be the vertex set. 
Recall that two graphs are isomorphic if there is an element in $\Sym_V$ that takes 
the incidence relation of one graph to the incidence relation of the other.
That is, for graphs $(V,R_1)$ and $(V,R_2)$ and for $g \in \Sym_n$, 
$$
(i^g,j^g) \in R_2 \iff (i,j) \in R_1.
$$
We let $E = {V \choose 2}$ be the set of all unordered pairs of elements of $V$ 
and we identify a graph with the subset of $E$ 
that corresponds to the incident pairs of vertices in the graph.
The lattice $\frakL$ and the partially ordered set $\cP$ are both equal to $\frakP(E)$, the power set of $E$.
Thus, the test function $\chi$ is trivial: it yields the value one for every $x\in \frakL.$
The ordering of graphs is given by the inclusion of the associated subsets of $E$.
Thus, 
a graph $(V,R_1)$ is contained in $(V,R_2)$ if $(i,j) \in R_1$ implies 
$(i,j) \in R_2$ for all $i,j \in V$.
Figure~\ref{g4full} shows the lattice  $\cP=\frakL$ of graphs on  
$n=4$ vertices. 
The orbits of $G = \Sym_4$ are indicated by grouping the vertices together. 
\begin{figure}
\begin{center}
\input GRAPHICS/graph_4_poset_full_lvl_6.tex 
\end{center}
\caption{\label{g4full}The lattice $\frakL= \cP$ of graphs on 4 vertices}
\end{figure}
Of course, the large number of nodes make this diagram somewhat hard to read.
Often, all we need to know is a representative of each orbit. 
Hence it is more instructive to draw the poset of orbits as in Figure~\ref{g4poset}.
\begin{figure}
\begin{center}
\input GRAPHICS/graph_4_poset_lvl_6.tex 
\end{center}
\caption{\label{g4poset}The poset of orbits}
\end{figure}
The nodes represent orbits of graphs, and one representative graph is shown in each node. 
This representative is in fact the lexicographically least graph in its orbit.

\bigskip



Let us look at an example where the test function is more interesting.
Suppose we want to classify cubic graphs on $n$ vertices. 
A graph is cubic if every vertex is incident with exactly three edges. 
For cubic graphs to exist, the number of vertices has to be even.
Again, we let $V = \{0,\ldots,n-1\}$ be the set of $n$ vertices, and we have  
$E = {V \choose 2}$ the set of unordered pairs of $V.$ 
Again $\cL = \frakP(E)$ the power set of $E$. 
The test function $\chi$ and the poset $\cP$ is defined as follows. 
If $x$ is a subset of $E$, we set $\chi(x) = 1$ 
if each vertex in $V$ 
is incident with {\em at most three edges} in $x.$
It is easy to check that this function $\chi$ is a test function. 
We let $\cP = \{ x \in \frakL \mid \chi(x) = 1 \}$. 
The search poset $\cP$ has elements of rank $r$ for $0 \le r \le t$ 
where $t = \frac{3n}{2}.$ 
Let us choose specific values. For $n = 6$, we find $t=9.$

\bigskip


Suppose we want to classify cubic graphs on $6$ vertices. 
The poset of orbits is shown in Figure~\ref{posetcubicsix}.
\begin{figure}
\begin{center}
\input GRAPHICS/graph_6_r3_poset_lvl_9.tex 
\end{center}
\caption{\label{posetcubicsix}The poset of orbits for 3-regular graphs on 6 vertices}
\end{figure}


\bigskip



This computation appears somewhat inefficient. It is better to observe 
that the complement of a cubic graph on $6$ vertices is regular of degree $2.$
The $2$-regular graphs on $6$ vertices are much easier to compute since they have only $6$ edges, and hence $t=6.$ 
For 2-regular graphs, 
we consider the search poset consisting of subsets of $E$ 
which have the property that 
each vertex is incident with {\em at most two edges}.
The resulting poset of orbits for this problem is shown in Figure~\ref{posetcubicsixcomplement}.
\begin{figure}
\begin{center}
\input GRAPHICS/graph_6_r2_poset_lvl_6.tex 
\end{center}
\caption{\label{posetcubicsixcomplement}The poset of orbits for 2-regular graphs on 6 vertices}
\end{figure}




%\begin{figure}
%\begin{center}
%\input GRAPHICS/graph_6_r3_poset_lvl_9.tex 
%\end{center}
%\caption{\label{posetcubicsix}The poset of orbits for cubic graphs on 6 vertices}
%\end{figure}
%The tree of orbits is shown in Figure~\ref{treecubicsix}.
%\begin{figure}
%\begin{center}
%\input GRAPHICS/graph_6_r3_tree_lvl_9.tex 
%\end{center}
%\caption{\label{treecubicsix}The tree of orbits for cubic graphs on 6 vertices}
%\end{figure}



\bigskip

Let us now look at an example involving subspaces.

\bigskip



Suppose we want to classify optimal linear codes. 
An $[n,k,d]_q$ code is a linear 
code over the field $\bbF_q$ of length $n$, 
dimension $k$ and with minimum distance $d.$ 
What this means is that the code is a vector subspace of $\bbF_q^n$ of dimension $k$, 
such that the minimum distance between any two vectors in the space is at least $d$ 
and there are two vectors at distance $d$. 
The distance between two vectors is the {\em Hamming distance,} 
which counts the number 
of places in which two given vectors differ. 
The most interesting codes 
have large $d$. 
So, for fixed $n$ and $k$ and $q$, we want to find the codes with largest $d$. 
These kinds of codes are called {\em distance optimal}. 
The group $G=\Gamma M(n,q)$ of semilinear monomial matrices over $\bbF_q$ 
acts on the set of codes 
and two codes are called equivalent 
if they lie in the same orbit under the group $G$.
Suppose we want to find the best codes of length $8$ and dimension $4$ over the field $\bbF_2$
Suppose that we want a minimum distance of $4$.
We model this problem using the lattice $\frakL$ of all subspaces of $\bbF_2^8.$ 
The search poset $\cP$ consists of all subspaces whose minimum distance is at least $4$. 
The classification algorithm produces the poset of orbits indicated in Figure~\ref{posetmonomialHamming}.
\begin{figure}
\begin{center}
\input GRAPHICS/Monomial_8_2_poset_lvl_4.tex 
\end{center}
\caption{\label{posetmonomialHamming}The poset of orbits for the classification of $[8,4]$ codes with distance at least $4$}
\end{figure}
Each level in this poset represents orbits of the monomial group on codes of length $8$ with a certain dimension.
The topmost level is dimension $0$, the bottom level is dimension $4$.
Since we require distance $4$, the choices for the first basis vector are limited to 
vectors with weight at least $4$. 
The monomial group is transitive on vectors of a fixed weight, so the orbits at level one correspond to 
subspaces spanned by the vectors 
$$
(1,1,1,1,0,0,0,0),
(1,1,1,1,1,0,0,0),
(1,1,1,1,1,1,0,0),
(1,1,1,1,1,1,1,0),
(1,1,1,1,1,1,1,1).
$$
Let us not go through all the other orbits except point out that the unique orbit at level $4$ (at the bottom) 
is the binary Hamming $[8,4,4]$ code with generator matrix
$$
\left[
\begin{array}{cccccccc}
1&1&1&1&1&1&1&1\\
1&1&1&1&0&0&0&0\\
1&1&0&0&1&1&0&0\\
1&0&1&0&1&0&1&0\\
\end{array}
\right]
$$
and automorphism group of order $1344$.
This computation is proof that the $[8,4,4]$ binary Hamming code is unique 
(since there is only one node at the bottom of this diagram).



\bigskip

The computation in this example is somewhat inefficient since there are a lot of orbits. 
An improvement can be achieved if the results described in~\cite{BettenCodes} are used. 
The idea is to replace the linear model with a suitable model of functions between sets.


\bigskip



Here is the result from~\cite{BettenCodes} that we need:
Under the mild assumption that $d$ is at least three, 
the combinatorial object can now be described using projective geometries.
Namely, it can be shown that a code over $\bbF_q$ 
of length $n$, dimension $k$ and distance at least $d$ 
corresponds to a set of $n$ points 
in $\PG(n-k-1,q)$ such that any $d-1$ are independent. 
Furthermore, 
equivalent codes under the action of $G$ correspond 
to projectively equivalent sets under $\PGGL(n-k,q)$ and conversely. 
For this reason, the problem of classifying optimal codes 
is equivalent to the problem of classifying up to projective equivalence
the sets of size $n$ in $\PG(n-k-1,q)$  
with the property that any $d-1$ are independent.

\bigskip

The problem of classifying sets of points in projective space can be modeled using 
functions between sets.
We consider 
subsets $x$ of $\PG(n-k-1,q)$ 
of size $n$ {\em or less} 
with the property that any $d-1$ are independent. 
If $x$ has size less than $d-1$, we require that $x$ itself be independent.
Let us call such subsets {\em special}.
Next, we consider the lattice $\frakL = \frakP(\PG(n-k-1,q))$ 
of all subsets of points of $\PG(n-k-1,q).$ 
We define a test function $\chi: \frakL \rightarrow \{0,1\}$ 
which maps $x$ to $1$ if $x$ is special and to $0$ otherwise. 
Then, as usual, $\cP = \{ x \in \frakL \mid \chi(x) = 1 \}$ is the search poset.

\bigskip

Let us look at  the example of binary $[8,4]$ codes with minimum distance $4$ again. 
We compute the poset of orbits 
of special subsets in $\PG(3,2)$ shown in Figure~\ref{fig:hamming}.
\begin{figure}
\begin{center}
\input GRAPHICS/Hamming.tex 
\end{center}
\caption{\label{fig:hamming}The poset of orbits for the Hamming code}
\end{figure}
The bottom node corresponds to the unique $[8,4,4]$ binary Hamming code.

\bigskip





The classification algorithm turns the poset of $G$-orbits into a tree.
We call this tree the {\em tree of $G$-orbits.}
The advantage of a tree is that each node and hence each $G$-orbit is visited exactly once.
The way that we more from the poset of $G$-orbits to the tree of $G$-orbits 
is by removing edges.
More precisely, we remove edges from the poset 
so that each node is incident with 
exactly one node below. 
This unique node below a given node is called 
the {\em canonical ancestor}.


\bigskip



The tree of orbits for the problem of 2-regular graphs from Figure~\ref{posetcubicsixcomplement} 
is shown in Figure~\ref{treecubicsixcomplement}.
\begin{figure}
\begin{center}
\input GRAPHICS/graph_6_r2_tree_lvl_6.tex 
\end{center}
\caption{\label{treecubicsixcomplement}The tree of orbits for 2-regular graphs on 6 vertices}
\end{figure}


\bigskip









