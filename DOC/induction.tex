

\subsection{The inductive step}
\label{subsec:inductivestep}



Let a group $G$ act on two finite set $X$ and $Y$. 
Let ${\mathcal R}$ be a $G$-invariant relation between $X$ and $Y.$ 
This means that 
$$
(x,y) \in {\mathcal R} \iff 
(x^g,y^g) \in {\mathcal R} 
$$
for all $g \in G$ and all $x \in X$ and $y \in Y$.

\bigskip

 
Let $p_1,\ldots,p_m$ be representatives for the $G$-orbits on $X$, with associated stabilizers $G_{p_i}.$ 
Let $q_1,\ldots,q_n$ be representatives for the $G$-orbits on $Y$, with associated stabilizers $G_{q_j}.$

\bigskip

We introduce the {\em projection mappings} $\Pi_i$ for 
cartesian products $S_1 \times S_2 \times \cdots \times S_k$ of sets. 
Namely, $\Pi_i$ is the projection onto the $i$-th component.
For $a \in X$, let $\Up(a) = \{ (x,y) \in {\mathcal R} \mid x = a \} = \Pi_1^{-1}(a).$
For $b \in Y,$ let $\Down(b) = \{ (x,y) \in {\mathcal R}  \mid y = b \}= \Pi_2^{-1}(b).$

\bigskip

For $i = 1,\ldots,m,$ let $m_i$ be the number of orbits of $G_{p_i}$ on $\Up(p_i)$, 
with representatives $u_{i,1}, \ldots, u_{i,m_i}.$
Let $\cU_{i,a} = u_{i,a}^{G_{p_i}}$ for $a=1,\ldots,m_i.$ 
Let $\frakU_i = \{ \cU_{i,a} \mid a=1,\ldots,m_i \}$ for $i=1,\ldots, m$ 
and put $\frakU = \bigcup_{i=1}^m \frakU_i.$

\bigskip

For $j = 1,\ldots,n,$ let $n_j$ be the number of orbits of $G_{q_j}$ on $\Down(q_j)$, 
with representatives $d_{j,1}, \ldots, d_{j,n_j}.$
Let $\cD_{j,b} = d_{j,b}^{G_{q_j}}$ for $b=1,\ldots,n_j.$ 
Let 
$\frakD_j = \{ \cD_{j,b} \mid b=1,\ldots,n_j \}$ for $j=1,\ldots,n$ 
and put $\frakD = \bigcup_{j=1}^n \frakD_j.$



\bigskip

The following basic result from~\cite{BettenBLT} 
for group actions on cartesian products is important. 
The origins of this lemma are somewhat unclear. 
Versions of this lemma with transitive group actions 
have appeared before, see Kerber~\cite[1.2.15]{Kerber99}.

\bigskip


\begin{lemma}\label{lem:UD}
Let $G$ be a group acting on finite sets $X$ and $Y$ and let  
${\mathcal R}$ be a $G$-invariant relation between $X$ and $Y.$ 
Let the orbit sets $\frakU$ and $\frakD$ be as above.
Then there exists a canonical bijection from $\frakU$ to $\frakD.$ 
\end{lemma}
\begin{proof}
The elements of $\frakU$ are in bijection to the orbits of $G$ 
on the flags (i.e., incident pairs) of the relation ${\mathcal R}.$ 
By symmetry, the elements of $\frakD$ are  in bijection to these orbits also. 
This gives the canonical bijection.\eop
\end{proof}

\bigskip

In what follows, we agree to call the bijection arising in Lemma~\ref{lem:UD} by the greek letter $\phi$.


\bigskip


Given an action of a group $G$ on a relation 
${\mathcal R} \subseteq  X\times Y.$
Suppose that we have solved the classification 
problem for $G$-orbits on $X.$
This means we know the orbits of $G$ on $X.$
We have chosen a set of orbit representatives 
$p_1, \ldots, p_m$ for the $G$-orbits on $X$ 
and we know the associated stabilizers $G_{p_i}$ for $i=1,\ldots,m.$ 
In addition, we assume that given an element $x \in X$, 
we can efficiently compute an element $g \in G$ and an integer $i \in \{ 1,\ldots,m\}$  such that $x^g = p_i.$ 


\bigskip

Our goal is to solve the classification problem for $Y$. 
This means, we want to choose representatives $q_1,\ldots,q_n$ for the $G$-orbits on $Y$ 
and compute the associated stabilizers. 
We also want to be able for any $y \in Y$ to efficiently compute 
an element $g \in G$ and an integer $j \in \{ 1,\ldots,n\}$  such that $y^g = q_j.$ 


\bigskip

We introduce a bipartite multigraph:

\bigskip

\begin{definition}\label{def-orbitgraph}
Let $G$ be a group acting on finite sets $X$ and $Y$ and 
let ${\mathcal R}$ be a $G$-invariant relation between $X$ and $Y.$ 
The {\em classification graph} is the bipartite graph
\begin{equation}\label{def:graph}
\Gamma_{\mathcal R}^G
\end{equation}
which has vertices corresponding to the orbits of $G$ on $X$ and the orbits of $G$ on $Y$.
%The vertices of type one are in bijection to the orbits of $G$ on $X.$ 
%The vertices of type two are in bijection to the orbits of $G$ on $Y.$ 
The edges of the graph correspond to the orbits of $G$ on the relation ${\mathcal R}.$ 
Incidence is natural. That is, a vertex $p_i^G$ is incident with the edge $(x,y)^G$ 
if $x \in p_i^G$. Likewise, the vertex $q_j^G$ is incident with the edge $(x,y)^G$ if 
$y \in q_j^G.$
The orbits of $G$ on $X$ form one bipartition, the orbits of $G$ on $Y$ form the other biprtition.
%More specifically, the edges between a pair of orbits $p^G$ of $G$ on $X$  and $q^G$ of $G$ on $Y$ 
%correspond to the  orbits of $G$ on the flags of the relation ${\mathcal R} \cap (p^G \times q^G).$
\end{definition}

From the lemma, it follows that the edges of $\Gamma_{\mathcal R}^G$ are in one-to-one correspondence to 
the elements of the set $\frakU$ and also the elements of the set $\frakD.$
The edges incident with vertex $p_i$ are in one-to-one correspondence to the elements of $\frakU_i.$ 
The edges incident with vertex $q_j$ are in one-to-one correspondence to the elements of $\frakD_j.$
Note that the classification graph is a multigraph by definition: 
There may be {\em several} edges connecting a given pair of orbits $p^G$ and $q^G.$ 


\bigskip

%\subsection{A Labeling Algorithm}
%\label{sec:basecase}



%\bigskip


The classification graph 
$$
\Gamma_{\mathcal R}^G
$$
can be utilized to lift the classification of $G$-orbits on $X$ to a classification of $G$-orbits on $Y$.
The only requirement is that each element $y \in Y$ 
is incident with at least one element from $X$. 

Once we can lift the classification as described above, 
we can classify orbits of groups on partially ordered sets.
We consider induction on the level sets
$\cP_0, \cP_1, \ldots, \cP_k$ of the poset. 
Starting with a classification of $G$-orbits on $\cP_0$ (which is often very easy to obtain), 
we can perform induction on the levels of the poset.
Assuming that the $G$-orbits on $\cP_i$ have been classified already, 
the procedure applied to the incidence relation 
between $\cP_i$ and $\cP_{i+1}$ will allow us to lift the classification from level $i$ to level $i+1.$ 
The assumption that each element in $\cP_{i+1}$ is incident with at least on element in $\cP_i$ 
is trivially satisfied if the $\cP_i$ are the level sets in the poset. 

  

\bigskip

Let us now describe the procedure to lift the classification.
Following Schmalz~\cite{Schmalz90}, we introduce the notion of a {\em Schmalz isomorphism}.
Using the notation introduced above, we assume that 
representatives $p_1,\ldots, p_m$ for the $G$-orbits on $X$ have been chosen.
We also assume temporarily that representatives $q_1,\ldots, q_n$ for the $G$-orbits on $Y$ are fixed.
In addition, we assume that the orbits $\frakU$ and $\frakD$ with the appropriate orbit representatives 
are known. Taking into account the canonical bijection $\phi: \frakU \rightarrow \frakD$
we consider a pair 
of orbits $\cU_{i,a} \in \frakU$ and $\cD_{j,b}\in\frakD$ with 
\begin{equation}\label{eqn:phi}
\cU_{i,a}^\phi = \cD_{j,b}.
\end{equation}
According to the proof of Lemma~\ref{lem:UD}, 
there exists an element $g \in G$ with 
\begin{equation}\label{eqn:schmalz}
u_{i,a}^g = d_{j,b},
\end{equation}
where $u_{i,a}$ is the representative for the orbit $\cU_{i,a}$ and 
$d_{j,b}$ is the representative for the orbit $\cD_{j,b}.$  
Whereas the bijection $\phi$ is canonical, the choice 
of the element  $g$ in~(\ref{eqn:schmalz}) is not. 
If $u \in \Stab_{G}(u_{i,a})$ and $v \in \Stab_G(d_{j,b})$ 
then $ugv$ satisfies $g$ whenever $g$ does.


\bigskip



The idea to store these group elements goes back to Schmalz~\cite{Schmalz90}.
For this reason, we decide to call the group element $g \in G$ satisfying~(\ref{eqn:schmalz}) 
the Schmalz isomorphism element for the pair $u_{i,a}$ and $d_{j,b}.$ 
An edge between $\cU_{i,a}$ and $\cD_{j,b}$ 
in $\Gamma_{\mathcal R}^G$ is said {\em to have been labeled} 
if the Schmalz isomorphism 
for $u_{i,a}$ and $d_{j,b}$ has been computed.
We say that the classification graph $\Gamma_{\mathcal R}^G$ {\em has been labeled}
if every edge has been labeled. 




\bigskip



Our classification algorithm can now be described. 
We have two main steps, called 
{\sc Lift} and {\sc Classify}. 
{\sc Classify} in turn invokes two algorithms, {\sc Define} and {\sc Eliminate}. 

\bigskip

In {\sc Lift}, we compute the sets $\Up(p_i)$ for all $i=1,\ldots,m.$ 
Once this is done, we compute the orbits of $G_{p_i}$ on $\Up(p_i)$ for all $i=1,\ldots,m.$ 
This gives representatives $u_{i,a}$ for orbits 
$\cU_{i,a}$ for $a=1,\ldots, m_i$ and all $i=1, \ldots, m.$ 
It also furnishes the stabilizer groups 
$G_{p_i,u_{i,a}} = \Stab_{G_{p_i}}(u_{i,a})$ for $a=1,\ldots, m_i$ and all $i=1, \ldots, m.$ 
Thus, the sets $\frakU_i = \{ \cU_{i,a} \mid a = 1,\ldots, m_i \}$ 
and $\frakU = \bigcup_i \frakU_i$ are known.


\bigskip

In {\sc Classify},
we loop over the set $\cC = \{ u_{i,a} \mid i=1,\ldots,m, \; a=1,\ldots, m_i \}$.
The strategy is to shrink the set $\cC$ down to a set $\cC'$ such that the 
transversal for the $G$-orbits on $Y$ 
is obtained by projecting members of $\cC'.$
Specifically, we will have  
$$\{q_1,\ldots,q_n\} = \{ \Pi_2(u) \mid u \in \cC' \}.
$$ 
The set $\cC'$ is never really formed. 
Instead, we perform a loop over the elements of the 
set $\cC$ 
and continuously remove elements from it until the set is empty.
For each element $u \in \cC$ picked, we invoke 
{\sc Define} and {\sc Eliminate} in turn.
In the beginning, we set $j = 1.$

\bigskip

Unless $\cC$ is empty, we pick the first element $u \in \cC$, remove it from $\cC,$ and execute 
{\sc Define} and {\sc Eliminate} in turn. 
Algorithm {\sc Define} puts
$q_j := \Pi_2(u).$
This defines our next representative of a $G$-orbit on $Y.$ 
We let $i$ and $a$ be so that $u = u_{i,a}.$ 
The group $G_{p_i,u_{i,a}} \le G_{q_j}$ is known from work done by {\sc Lift}.

\bigskip


The algorithm {\sc Eliminate} will remove all isomorphic copies 
of $u$ from $\cC$ and compute the group $G_{q_j}$ from the subgroup $G_{p_i,u_{i,a}}.$ 
For each element $(z,q_j) \in  \Down(q_j)$, we perform the following algorithm:
Determine an element $h_1 \in G$ with $z^{h_1} = p_k$ 
for some $k\in\{1,\ldots,m\}$ (depending on $z$). This can be done by assumption. 
Then compute an element $h_2 \in G_{p_k}$ such that $(z,q_j)^{h_1h_2}=(p_k,q_j^{h_1})^{h_2} = u_{k,c}$ 
for some $c \in \{1,\ldots,m_k\}.$ 
This element $h_2$ can be computed easily from the information stored by {\sc Lift}. 
If $k=i$ and $c=a,$ we find that $h_1h_2$ is a  representative 
for a coset of the subgroup $G_{p_i,u_{i,a}}$ of $G_{q_j}$ 
and we save it. 
Otherwise, we delete $u_{k,c}$ from $\cC$ and label the edge in 
$\Gamma_{\mathcal R}^G$
originating in $u_{k,c}$ by the Schmalz isomorphism $(h_1h_2)^{-1}.$
Once all elements of $\Down(q_j)$ have been processed, 
the coset representatives saved form a complete set of coset representatives 
of $G_{p_i,u_{i,a}}$ in $G_{q_j}$ and hence $G_{q_j}$ is known. 
Once all elements in $\Down(q_j)$ have been processed, 
{\sc Eliminate} is done. 

\bigskip

The algorithm {\sc Classify} then replaces $j$ by $j+1.$ Unless $\cC$ is empty, 
another round of {\sc Define} and {\sc Eliminate} is performed. 
Otherwise, {\sc Classify} puts $n:=j$ 
and terminates.


\begin{theorem}
Given an action of a group $G$ on a relation 
$$
{\mathcal R} \subseteq  X\times Y
$$
such that every $y \in Y$ is incident with at least one $x \in X.$ 
Given a transversal $p_1,\ldots,p_n$ of the $G$-orbits on $X$ with associated stabilizers. 
Assume that constructive recognition is available.
Then algorithms {\sc Lift} and {\sc Classify} 
compute the sets $\frakU$ and $\frakD,$
a transversal $q_1,\ldots,q_n$  for the $G$-orbits on $Y$ and
associated stabilizer groups $G_{q_j}$ for all $j=1,\ldots,n.$ 
They also compute the classification graph $\Gamma_{\mathcal R}^G$ and label the edges.
\end{theorem}
\begin{proof}
For the sake of simplicity, we denote the classification graph $\Gamma_{\mathcal R}^G$ 
as $\Gamma.$

\smallskip


Consider an element $y \in Y.$ 
We need to show that there is a group element $g \in G$ 
with $p^g = p_j$ for some $j \in \{ 1,\ldots,n\}$ (depending on $y$).
By assumption, there is at least one $(x,y) \in \Down(y).$ That is, 
there is at least one $x \in X$ with $(x,y) \in {\mathcal R}.$
Pick one such pair $(x,y) \in \Down(y)$.
By assumption, we find a group element $h_1 \in G$ 
with $x^{h_1} = p_i$ for some $i \in \{ 1,\ldots,m\}.$ 
From the work done by {\sc Lift}, we can find a group element $h_2 \in G_{p_i}$ with
$(x,y)^{h_1h_2} = u_{i,a}$ for some $a \in \{ 1,\ldots,m_i\}.$ 
Since the edges in $\Gamma$ have been labeled, 
we can retrieve the Schmalz isomorphism $h_3 \in G$ 
associated to the edge incident with $u_{i,a}$. 
Then $(x,y)^{h_1h_2h_3} = u_{i,a}^{h_3}=d_{j,b}$ for some $j \in \{ 1,\ldots,n\}$ 
and some $b \in \{1,\ldots,n_j\}.$ 
Thus, with $g = h_1h_2h_3$ we have $y^g = \Pi_2(d_{j,b}) = q_j$ for some $j \in \{ 1,\ldots,n\}.$

\smallskip

Let us consider the stabilizer of $q_j$ in the transversal computed by the algorithm.
Since $q_j = \Pi_2(u_{i,a})$ for some $i \in \{1,\ldots,m\}$ and $j \in \{ 1,\ldots,m_i\},$ 
we have the group 
$$
H := G_{u_{i,a}} = \Stab_{G_{p_i}}(u_{i,a}) \le G_{p_i}  
$$
computed inside  {\sc Lift}. 
But
$$
H = G_{q_j} = G_{d_{j,b}} \le  G_{q_j} 
$$ 
for some $(j,b)$ also. 
The cosets of $H$ in $A = G_{q_j}$ correspond to the different images of 
$u_{i,a} = d_{j,b} \in \Down(q_j).$ 
Algorithm {\sc Eliminate} loops over all elements in $\Down(q_j).$ 
So, if a pair $(z,q_j) \in \Down(q_j)$ exists which is an image of $u_{i,a}$ 
under an element $a\in A$, then {\sc Eliminate} will process 
this element and will find a pair $(k,c)$ and group elements $h_1h_2$ 
with $(k,c) = (i,a)$ and  
$$
(z,q_j)^{h_1h_2}= (p_k,q_j^{h_1})^{h_2} = u_{k,c} = u_{i,a}
$$
This $h_1h_2$ is a coset representative for $H$ in $A$. 
The group $A$ is formed by extending $H$ by all these coset representatives. 
Hence, upon termination of {\sc Eliminate}, we have $A = G_{q_j}.$ 


\smallskip

The set $\frakD$ is computed implicitly. 
For each $j = 1,\ldots,n,$ consider all edges $u_{i,a} \in \Gamma$ such that 
$u_{i,a}$ is incident with $q_j^G.$ 
The set of all images $q_{i,a}^g,$ where $g$ 
is the Schmalz isomorphism associated to the edge $u_{i,a}$ 
forms a system of orbit representatives 
for the orbits of $G_{q_j}$ on $\Down(q_j).$ 
Hence $\frakD_j$ is known.
Repeating this for $j=1,\ldots,n,$ yields the set $\frakD = \bigcup_{j=1}^n \frakD_j.$
\eop
\end{proof}







