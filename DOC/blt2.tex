\section{Classifying BLT-Sets By Breaking the Symmetry}


If we want to proceed to higher orders of $q$, we find that 
the classification method in \verb'blt.out' is not optimal.
For this reason, we resort to the method of Breaking the Symmetry.
This means the following. 
In Step 1, we classify the partial BLT-sets of some size. Let us call the representatives of the isomorphism types starter. 
In Step 2, we lift -- in all possible ways -- all starter sets to BLT-sets. 
This can be done by formulating the problem of lifting BLT-sets 
in the language of graph theory.
The lifts of a BLT-starter $S$ are in one-to-one correspondence to the 
rainbow cliques of size $q+1-|S|$ in a certain colored graph $\Gamma_{S,\ell}.$ 
Here, $\ell$ is a line intersecting $S$ in a single point.
We will have \verb'blt.out' create the colored graphs, 
and invoke the program \verb'all_cliques.out' to do the clique finding for us.
Computationally, the clique finding step is the dominant part of the algorithm.
Once all the cliques of $\Gamma_{S,\ell}$ for all starter sets $S$ 
have been found, we move on to Step 3, where we do a final isomorph classification.
This is done back in \verb'blt.out'.


\bigskip

Let us look at an example. Suppose we wish to classify the BLT-sets 
of $Q(4,23).$
The example \verb'makefile' for this classification problem 
is in 
\begin{quote}
\verb'cd ORBITER/DATA/BLT/23'\\
\end{quote}
In this directory, we issue the command 
\verb'make'.
This will compute the classification of BLT-sets of order $q=23$. 
After a few minutes, the file 
\begin{quote}
\verb'ORBITER/DATA/BLT/23/report_BLT_23.pdf'
\end{quote}
containing a report of the $9$ BLT-sets of $Q(4,23)$ is created. 
This file is human-readable. 
%The content of this report 
%should agree with the information that is available from the web site
%$$
%\mbox{http://www.math.colostate.edu/$\sim$betten/blt.html}
%$$
To change the parameter $q$ to any odd prime power $q$ you wish to investigate, 
we can change the line
\begin{quote}
\verb'Q=23'
\end{quote}
in \verb'makefile'. Larger values of $q$ require longer computing times. 
The largest value of $q$ that has been 
run to this day was $q=67$. 
The computation was performed in parallel and took $16$ years 
total to complete. This is definitely not a computation 
that should be tried at home.

